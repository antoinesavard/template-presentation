\documentclass[thicklines,fleqn,xcolor=dvipsnames,compress,12pt,aspectratio=169]{beamer}
\mode<presentation>
\usepackage[english]{babel}
\usepackage[T1]{fontenc}
\usepackage[utf8]{inputenc}
\usepackage{graphicx}
\usepackage[export]{adjustbox}
\usepackage{svg}
\usepackage{IEEEtrantools}
\usepackage{physics,dsfont,mathrsfs,cancel,tensor,slashed,mathtools}

\usepackage{multicol}
\usepackage{amsmath}
\usepackage{amsfonts}
\usepackage{eulerpx}
\usepackage{bm}
\usepackage{setspace}
\usepackage{hyperref}
\usepackage{subcaption}
\usepackage{mwe}

\usepackage[round]{natbib}

\usepackage{textpos}
\usepackage{multirow}% http://ctan.org/pkg/multirow
\usepackage{hhline}% http://ctan.org/pkg/hhline
\usepackage{tikz}
\usetikzlibrary{arrows.meta}

\hypersetup{
    colorlinks,
    citecolor=,
    linkcolor=black,
    urlcolor=blue
}

\usetheme{default}
\newcommand\mycolor{PineGreen} % change the accent color here https://en.wikibooks.org/wiki/LaTeX/Colors#:~:text=black%2C%20blue%2C%20brown%2C%20cyan,be%20available%20on%20all%20systems.
\usefonttheme[]{professionalfonts} %font
\useinnertheme{rectangles}
\setbeamertemplate{navigation symbols}{} 
\setbeamertemplate{blocks}[rounded]
\setbeamertemplate{footline}

\setbeamertemplate{footline}
{
  \hbox{\begin{beamercolorbox}[wd=1\paperwidth,ht=2.25ex,dp=1ex,right]{framenumber}%
    \usebeamerfont{framenumber}\insertframenumber{}/\inserttotalframenumber\hspace*{2ex}
    \end{beamercolorbox}}%
  \vskip0pt%
}

% title
\setbeamerfont{title}{size=\huge}
\setbeamercolor{title}{fg=black}

% frametitle
\setbeamerfont{frametitle}{size=\Large}
\setbeamercolor{frametitle}{fg=black}

% toc
\setbeamerfont{section number projected}{size=\large}
\setbeamercolor{section number projected}{bg=\mycolor,fg=white}
\setbeamercolor{section in toc}{fg=black}
\setbeamercolor{subsection in toc}{fg=gray}
\setbeamertemplate{section in toc}[square]
\setbeamertemplate{subsection in toc}[square]

% bullets in itemize
\setbeamercolor{itemize item}{fg=\mycolor}
\setbeamertemplate{itemize item}[square]
\setlength{\leftmargini}{0.5cm}

% blocks
\setbeamercolor{block title}{bg=gray!20,fg=black}

% color boxes
\newcommand*{\colorboxed}{}
\def\colorboxed#1#{%
  \colorboxedAux{#1}%
}
\newcommand*{\colorboxedAux}[3]{%
  % #1: optional argument for color model
  % #2: color specification
  % #3: formula
  \begingroup
    \colorlet{cb@saved}{.}%
    \color#1{#2}%
    \boxed{%
      \color{cb@saved}%
      #3%
    }%
  \endgroup
}

% color cancel
\newcommand{\Xcancel}[2][]{%
  \ifblank{#1}{}{%
    \renewcommand{\CancelColor}{#1}%
  }
  \xcancel{#2}% 
}

\usepackage{pgfkeys}
% Définition des nouvelles options xmin, xmax, ymin, ymax
% Valeurs par défaut : -3, 3, -3, 3
\tikzset{
xmin/.store in=\xmin, xmin/.default=-3, xmin=-3,
xmax/.store in=\xmax, xmax/.default=3, xmax=3,
ymin/.store in=\ymin, ymin/.default=-3, ymin=-3,
ymax/.store in=\ymax, ymax/.default=3, ymax=3,
}
% Commande qui trace la grille entre (xmin,ymin) et (xmax,ymax)
\newcommand {\grille}
{\draw[help lines] (\xmin,\ymin) grid (\xmax,\ymax);}
% Commande \axes
\newcommand {\axes} {
\draw [>=stealth,->] (\xmin,0) -- (\xmax,0);
\draw [>=stealth,->] (0,\ymin) -- (0,\ymax);
}
% Commande qui limite l’affichage à (xmin,ymin) et (xmax,ymax)
\newcommand {\fenetre}
{\clip (\xmin,\ymin) rectangle (\xmax,\ymax);}

% fleqn dans IEEE
\IEEEeqnarraydefcolsep{0}{\leftmargini}

%----------------
% title page
%----------------

\title{Big important title}
\author{Foo Bar}
\date{Where --- When}

\begin{document}

%----------------------------------------------------------------------------------------
% TITLE
%----------------------------------------------------------------------------------------


\begin{frame}{}
    \maketitle
    
\end{frame}

%----------------------------------------------------------------------------------------
% TABLE OF CONTENTS
%----------------------------------------------------------------------------------------

\begin{frame}{Contents}
    
    \tableofcontents[hideallsubsections]
    
\end{frame}

%----------------------------------------------------------------------------------------
% INTRO
%----------------------------------------------------------------------------------------

\section{Introduction}

\begin{frame}
    
    \begin{columns}
        \column{0.2\textwidth}
        \column{0.75\textwidth}
            {\Large Introduction}
    \end{columns}
    
\end{frame}

\begin{frame}{Why should we\\ listen to this}
    
    \begin{columns}
        \column{0.2\textwidth}
        \column{0.75\textwidth}
            {\large Explanation 1.}\\
            {\color{gray} Precision 1.}\\\vspace{1em}
            {\large Explanation 2.}\\
            {\color{gray} Precision 2.}\\\vspace{1em} 
            {\large Explanation 3.}\\
            {\color{gray} Precision 3.}
            
    \end{columns}
    
\end{frame}

\begin{frame}{Figure\\description}
    
    \begin{columns}
        \column{0.2\textwidth}
        \begin{itemize}
            \item point 1
            \item point 2
            \item point 3
        \end{itemize}
        \column{0.75\textwidth}
            \begin{figure}
                \includegraphics[width=0.7\linewidth]{example-image-a}\\
                {\tiny Credit: bar}
            \end{figure}
    \end{columns}
    
\end{frame}

\begin{frame}{Example\\of items}
    
    \begin{columns}
        \column{0.2\textwidth}
        \column{0.75\textwidth}
        \begin{itemize}
        \setlength\itemsep{1em}
            \item {\large item 1}
            \item {\large item 2}
            \item {\large item 3}
        \end{itemize}
    \end{columns}
    
\end{frame}

%----------------------------------------------------------------------------------------
% MODEL
%----------------------------------------------------------------------------------------

\section{Section}

\begin{frame}
    
    \begin{columns}
        \column{0.2\textwidth}
        \column{0.75\textwidth}
        
            {\Large Section title}
        
    \end{columns}
    
\end{frame}

% some math with appearing slides

\begin{frame}{some math with\\appearing slides}
    
    \begin{columns}
        \column{0.2\textwidth}
        \begin{figure}[h!]
        \centering
            \begin{tikzpicture}[xmin=-3,xmax=1,ymin=-3,ymax=1,scale=0.70]
                \axes
                \draw [line width=1.2pt] [domain=0:2*pi,scale=0.65] plot[smooth] ({-2+2*cos(\x r)-sin(\x r)},{-2+sin(\x r)+2*cos(\x r)});
                \draw [line width=1.2pt,dotted] [domain=0:2*pi,scale=0.65] plot[smooth] ({-2+cos(\x r)-1/2*sin(\x r)},{-2+1/2*sin(\x r)+cos(\x r)});
                \draw [scale=0.65] (-2,-2) node {$\bullet$};
                \draw [scale=0.65] (-4,-4) node {$\bullet$};
                \draw [scale=0.65] (-3,-1) node {$\bullet$};
                \draw [scale=0.65] (-1,-3) node {$\bullet$};
                \draw [scale=0.65] (-4.2,-4) node[below] {$C$};
                \draw [scale=0.65] (-3,-0.8) node[left] {$S$};
                \draw [scale=0.65] (-1,-3.2) node[right] {$S$};
                \draw [scale=0.65] (-0.1,-2) -- (0.1,-2) node[right] {$-\frac{P}{2}$};
                \draw [scale=0.65] (-2,-0.1) -- (-2,0.1) node[above] {$-\frac{P}{2}$};
                \draw (1,0) node[right] {$\sigma_1$};
                \draw (0,1) node[above] {$\sigma_2$};
            \end{tikzpicture}
    \end{figure}
        \column{0.75\textwidth}
            \onslide<1->
            \begin{flalign}
                    m\left[\pdv{\bm{u}}{t} +\right. \overbrace{\Xcancel[\color{Goldenrod}]{(\bm{u}\cdot\grad)\bm{u}}}^{\mathclap{\text{Advection}}}\bigg] = -\underbrace{mf\>\bm{\hat k}\cross\bm{u}}_{\text{Coriolis}} + \overbrace{\bm{\tau}_\text{a} + \bm\tau_\text{w}}^\text{Forcing} - \underbrace{mg\grad H_\text{d}}_{\mathclap{\text{Sea surface height}}} + \overbrace{\div\bm{\sigma}}^{\mathclap{\text{Rheology}}}\nonumber
            \end{flalign}
            %
            \onslide<2->
            \begin{equation}
                \sigma_{ij} = 2\eta\dot\varepsilon_{ij} + (\zeta-\eta)\dot\varepsilon_{kk}\delta_{ij} - \frac{\colorboxed{Goldenrod}{P}}{2}\delta_{ij},\quad \zeta = \frac{\colorboxed{Goldenrod}{P}}{2\Delta}, \quad \eta  =  \frac{\zeta}{e^2} \nonumber
            \end{equation}
            %
            \onslide<3->
            \begin{equation}
                P = P^*\colorboxed{Goldenrod}{h}\exp\left\{-C(1-\colorboxed{Goldenrod}{A})\right\}\nonumber
            \end{equation}
    \end{columns}
    
\end{frame}

%----------------------------------------------------------------------------------------
% METHODS
%----------------------------------------------------------------------------------------

\section{Section subsection}

\begin{frame}
    
    \begin{columns}
        \column{0.2\textwidth}
        \column{0.75\textwidth}
            \vspace{5em}\\
            {\Large Section title}\vspace{1em}
            \begin{itemize}
                \item Subsection 1
                \item Subsection 2
                \item Subsection 3
            \end{itemize}
            
        
    \end{columns}
    
\end{frame}

\begin{frame}{Subsection 1}
    
    \begin{columns}
            
        \column{0.2\textwidth}
        \vspace{1em}\\
            \color{Goldenrod}{Text 1}\\\vspace{1em}
            Text 2\\\vspace{1em}
            Text 3
        \column{0.75\textwidth}
            \begin{IEEEeqnarray}{0rCl}
                y &=& ax + b,\nonumber\\
                y &=& ax + b,\nonumber\\
                y &=& ax + b.\nonumber
            \end{IEEEeqnarray}
        
    \end{columns}
    
\end{frame}

\begin{frame}{Other\\ configuration.}
    
    \begin{columns}
            
        \column{0.2\textwidth}
        
            \begin{figure}
                \includegraphics[width=\textwidth, left]{example-image-b}
            \end{figure}
        
        \column{0.75\textwidth}
            
            {\large Text 1}\\
            {\color{gray} Precision 1.}\\\vspace{1em}
            {\large Text 2}\\
            {\color{gray} Precision 2.}\\\vspace{1em} 
            {\large Text 3}\\
            {\color{gray} Precision 3.}
         
    \end{columns}
    
\end{frame}


\begin{frame}
    
    \begin{columns}
        \column{0.2\textwidth}
        \column{0.75\textwidth}
            {\Large Thank you!}

    \end{columns}
    
\end{frame}



\end{document}















